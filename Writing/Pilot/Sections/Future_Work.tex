Ongoing work will continue to explore the best ways in which markets can be created. This includes consideration of partial matches or lotteries in MIP (Roth 1993). One may also consider the ability to leverage wagering in which job seekers would have an allocated number of “points” they could allocate amount the various potential jobs and vise versa for job wonders. \cite{2011_Budish} Additional research would include how to incorporate synergy of selection preferences by Job owners if allowed more than one person \cite{1985_Roth_b}.  Finally, since some of our methodology discovered many of our considered markets have an “on-line” placement procedure, exploration of how to best align preferences in such markets would require further exploration. For this work we do not compare to online job placement \cite{1994_Khuller}, such as that which happens at many information-sensitive commands due to the trickle of clearance issuance as opposed to bulk assignment. Forthcoming work on verifying and validating our implementation of inferred preferences will continually be bolstered as well \cite{2019_Shaw}.


% This is where our future work will go.

% \begin{enumerate}

% \item Explore implications of a ``Separation'' ($u$) preference

% \item Explore strategic Importance of positions. This can either be an iteration where billets in priority tranche's are run iterative (tier 1 billets all matched, those sailors and billets are taken out of the pool, then tier 2 billets all matched, etc.). Or a weighting where in a single matching optimization the preferences of higher tiered billets are given a greater weighting

% \item Does adding weight for specialization I’m objective function help the Navy better? Did job owners who wanted specialized sailors get them? Did sailors who wanted specialized jobs get them?

% \item Tailored Compensation decisions. 

% \item Incorporate timeliness of expected rotation date for availability windows.

% \item Multiple firms and multiple workers, each sailor can opt into collaterals and other roles \cite{1982_Roth}

% \item K-Means Clustering Analysis (or some other unsupervised machine learning) of preferences  

% \end{enumerate}

% A possible move away from the approach of mixed-integer programming with ordinal preferences would be to explore the viability of Budish's wagering formulation of approximate competitive equilibrium from equal incomes. \cite{2011_Budish}.