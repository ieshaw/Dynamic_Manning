We quite notably solved a problem previously unexplored in mathematical and economic academia. That problem is ‘implied preferences,’ the notion that a market participant can only rank so many jobs in her day-to-day as a human.  To remedy the vast volume issue, we mathematically predict preference rankings based on previous choices without the need for AI/ML.  In the future, extensive personnel data will enable deep learning insights to improve upon this implied preference algorithm. 

An important note is the attempt at strategy proofness of the algorithm \cite{2011_Budish} \cite{2006_Atila}, thus creating an incentive for honesty \cite{1982_Roth}. It is necessary to ensure that there is no benefit for a participant to provide incomplete preferences.  Likewise, we must acknowledge that there can be difficulty associated with a market participant ranking an incredibly large number of opportunities. To complete this necessity, we leverage the completion of incompletely provided preferences, the aforementioned ‘implied preference’ solution \cite{2019_Shaw}. 

The complete strategy proofness investigation of this formulation is left as an aspect of future work, but we acknowledge that there could be benefits to coalitions or awareness of the preference landscape of competitors. There is also a possibility of rejecting the implied preferences, as one market maker concluded unexpressed preferences indicate indifference \cite{1994_Irving} -- an expression of preference all its own. Nevertheless, the strategy proofness of Irving’s approach in an MIP matching process has yet to be explored. 
