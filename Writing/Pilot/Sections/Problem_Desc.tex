Human talent allocation is a resource intensive process for an institution. This paper outlines the mathematics necessary to lay the groundwork for such a task within one of the most complex institutions of the American psyche -- the Department of Defense (DoD).  We dove into this problem with a specific service branch in mind.  Focusing our work on the Navy, and within it, the sailors assigned to the National Security Agency (NSA) for cryptologic and cyber warfare efforts, we honed our test case to a group of officers most aligned with technical advancement. Furthermore, despite the small prototype sample, the algorithm is greatly scalable. 

Market forces appear are largely the drivers behind a job marketplace -- this concept is not new for the American economy.  However, given a captive audience such as service members ordered to specific jobs, the mathematical benefit of benevolent autocracy is immense.  Rather than the suboptimal fate to which current servicemembers are subjected, the DoD is shopping for a better way forward.  

In the Navy, terminology of a ‘detailing marketplace’ refers to the attempts to ‘detail,’ or place, warfighters into their next job.  This detailing marketplace is a system by which military members can transparently rank their job preferences and the people who ‘own’ those jobs can supply their preference of incoming personnel. This problem is rather unique to the military due to the aforementioned captive market of many members obligated to remain in service due to contract, desiring to stay in for a pension, wishing to stay in through a sense of service, complemented by an inability for lateral entry -- almost all members needing to start from the entry level.  Additionally, servicemembers change jobs nearly every one to three years. 

There are ontological parallels to other communities, too. This process is shared in some regards by the medical school graduates applying to the residency stage of their training. These graduates are applying to a pool of US residency programs. This pool is also rather narrow, mostly coming from US based medical schools, and the specificity of skill set required for success is better understood than success as a military officer. 

This general similarity is the reason our initial matching algorithm (Gale-Shapely Deferred Acceptance Algorithm) is similar one used by that process, the National Residency Match Program. The Gale-Shapely algorithm's application to the residency matching problem earned the Nobel Prize in 2012.

This paper also proposes an optimization based solution to the matching process. The optimization is not found in the National Residency Match Program as medical students have the ability to reject their assigned position, a choice not always given to military members. Thus an optimization can often more optimal solution for the DoD system at the expense of a few forced members.
 
Due to the importance of co-locating dual-military households (where two family members are in the military), we focus our matching algorithm and optimization proposals on solutions that would guarantee 95\% or greater co-location rate.

Acknowledging the difficulty of wrangling disparate and dated personnel data, this paper also explores helpful metrics that can be gleaned simply from submitted, ordered preferences by job seekers and job owners. These are competitiveness, similarity, generalism, and specialization. Interestingly, due to the fact that preferences are expressed on job seekers and the jobs themselves, these metrics can be developed about the jobs or the job seekers.
 
Further the paper ends with suggested metrics that would be gleaned if personnel data beyond preferences was accessible, clean, and structured. These include a similarity measure based on quality encodings and a suggested ordering of possible jobs or applicants. The latter is proposed to be enabled by deep learning (the underlying technology of Artificial Intelligence (AI)). The suggested ordering would not make decisions on placement, but rather provide job seekers and job owners with metrics distilling the vast amount of information about
