Similarity
Definition.
Similarity indicates how closely the preferences of two Sailors or two job owners’ preferences are aligned.

[math here]


Impact. 
High similarity scores between pairs of sailors or pairs of job owners may be used to identify hidden groups; clusters of skills or jobs that are not clear to the naked eye. It may reveal unseen jobs or sailors, creating a more perfect information space. For instance, imagine two sailors, A and B, who have a high similarity score. If Sailor B did not know about a job, X, or did not rank that job for another reason, the similarity score may alert Sailor B that they are a good fit for that job if they did not receive any of their other preferences.

To understand the multi-dimensional similarity of two vectors, the natural choice is the squared Euclidean distance between the two vectors. 

\[\sum_{k=1}^{m} (P^S_{i,k} - P^S_{j,k})^2\]

Yet, a job seeker's first preference is much more important to them than their hundredth preference, indicating that the similarity of top preferences of two job seekers is indicative of their overall similarity than the comparison of their much lower ranked preference.  Thus we want a weight on the distance according to the average importance of the metric to the two seekers.

\[w_k = \frac{2m - (P^S_{i,k} + P^S_{j,k})}{2m}\]


Thus we propose a similarity measure of two job seekers using preference data should be a weighted squared Euclidean distance between their two preference vectors. To ensure the maximum value is 1 for complete similarity (identical preferences), we scale the metric by a factor of $(m-1)^2$ (because the maximum difference between two rankings is $m-1$)and subtract it from 1.

Thus the similarity function is
\[\texttt{Similarity}(P^S_i, P^S_j) = 1 - \frac{1}{(m-1)^2}\sum_{k=1}^{m} \frac{2m - (P^S_{i,k} + P^S_{j,k})}{2m}(P^S_{i,k} - P^S_{j,k})^2\]