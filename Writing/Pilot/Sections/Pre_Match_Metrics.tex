Competitiveness
Definition.
Competitiveness measures the relative desirability of a given sailor or job based on the expressed preference ranking of the other party. We wanted to consider a more sophisticated way of determining this metric beyond an average ranking for either the sailor or job as it is very susceptible to a right tail bias.  For example, a position ranked 1st by ten individuals and hundredth by twenty individuals has the same average ranking as a job ranked seventh by all thirty individuals; yet the prior is much more competitive. To attempt to compensate for this, we adjust the average by a power of one half is to lessen the impact of very low preferences thereby weighting favorable preference more in our score consideration. 
 In order to generalize the competitiveness score we scale the resulting average by the total number of jobs or sailors within the system. This allows for competitiveness to range from 1 being most competitive, and 0 being not competitive at all.


[math here]


Impact.
Competitiveness will be used to rapidly identify the most attractive candidates. (E.g. Sailor A has a 0.98 competitiveness score and is therefore a top ranked candidate amongst all job owners.) These candidates would be good fits in many jobs and therefore could be considered ideal individuals to screen for command. On the job side, a low competitiveness would be a great way to decide on incentive structures. These jobs are those that very few sailors prefer and therefore either promotion based assignment or additional compensation could to targeted towards these positions. 

Specialization
Definition. 
Specialization measures the extent to which the maximum desire for a sailor or job differs from the mean. Sailor A is highly specialized if one job owner highly desires them much more than the mean job owner (e.g. Sailor A is the 1st preference choice of Job Owner X, but is on average the 21st ranked preference).

[math here]

Impact. 
A sailor with a high specialization has some skill set that will serve a specific job function well; the seeker has an advantage over peers for a single role compared to the other roles the seeker is qualified for. If most job owners rate a seeker with low preference but a single owner rates that seeker with a high preference, the marginal gains of that seeker being matched with that owner are high. Matching the sailor with this particular role utilizes their specialization and produces a disproportionate positive contribution to the Department of the Navy. Intuitively, the matching algorithm will favor creating high-specialization matches. Failing to make these matches has a more negative impact on the system than failing to make non-specialized matches, because the usefulness of the sailor goes down significantly between their optimal match and average matches. This metric may be useful in the future to examine if very particular aspects of jobs or sailors make them attractive to small subsets of the other population.


Preference Correlation
Definition.
Preference correlation is the R-squared score of job owner and job seeker preferences.  A strong positive correlation indicates that job owners and seekers mutually desire each other. 

[example plot + line...strong correlation]

[example plot + line...weak correlation]

Impact. 
A lack of correlation indicates an information gap between job seekers and owners; seekers may not understand what skills a particular job requires or job owners may lack an understanding of what skills would be beneficial to have on their team. 

Similarity
Definition.
Similarity indicates how closely the preferences of two Sailors or two job owners’ preferences are aligned.

[math here]


Impact. 
High similarity scores between pairs of sailors or pairs of job owners may be used to identify hidden groups; clusters of skills or jobs that are not clear to the naked eye. It may reveal unseen jobs or sailors, creating a more perfect information space. For instance, imagine two sailors, A and B, who have a high similarity score. If Sailor B did not know about a job, X, or did not rank that job for another reason, the similarity score may alert Sailor B that they are a good fit for that job if they did not receive any of their other preferences.
