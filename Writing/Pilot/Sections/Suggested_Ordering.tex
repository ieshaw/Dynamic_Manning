\subsection{Suggested Ordering}

If data is managed and stored properly it could be used to train a machine learning algorithm to predict the optimal ordering for a job seeker and for a job owner. These metrics would not be used to set the matching, but rather suggest to the seeker and owner what the best preference ranking would be based on historic tendencies. This could be useful especially if job seeker is asked to rank on the order of a hundred or more jobs, or if the billet owner has a hundred or more applicant seekers. This suggested ordering would provide a more holistic default than either no ranking, an arbitrary ranking such as alphabetically, or a single metric ranking such as PT score.

The previous data from preferences, performance evaluations, and other qualities in the personnel data (if managed accurately and structured properly) could all be taken into account to provide a holistic assessment beneficial to both sides. Essentially the computer could take in all the information and distill it to a single metric in a more consistent and timely manner than the humans who are already overwhelmed by the other demands of the job seeking/hiring process.