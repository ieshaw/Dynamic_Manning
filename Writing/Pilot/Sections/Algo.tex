A basic algorithm, like Gale-Shapely’s set of instructions for stable marriages, is an important first step in talent allocation. We moved onto a linear programming concept, oriented around solvers -- algorithms that are adaptable.  These solvers have varying types like linear, convex, and mixed integer.  Our group followed a path paved by Stanford operations researcher Alvin Roth and adapted a mixed integer solution to leverage its flexibility.  While Gale-Shapely’s algorithm can be solved in n-squared time, ours is n-p-complete, indicating polynomial time completion. What we lose in speed, we gain in the ability to alter the formulation without total overhaul.  These qualities make the approach particularly conducive to constraint creation, a vital requirement for stakeholders utilizing our solution.  

Ours is a unique subset of mixed integer programming (MIP), binary optimization, to reflect the nature of either placement in a specific job, or not (i.e. only 1 or 0). The resulting lattice has an incredibly large number of dimensions manifesting every potential job placement for every individual.  Subsequently, we apply constraints -- the inherent reason we utilize MIP.  These constraints cut away pieces of the lattice to reveal a viable search space to run the minimization function.  Finding the minimum point within that constrained lattice provides us with a matrix of the optimally matched jobs and sailors.

The constraints we apply are the same as the college admissions problem \cite{1985_Roth}, where each job can fill up to but not exceed the allocated number of positions and each sailor can only be assigned to one position. The explicit program can be found in Appendix \ref{Matching}. 