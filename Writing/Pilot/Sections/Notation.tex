
Throughout the paper, we will reference the notation listed in this section.

\begin{align}
m &= \text{number of different jobs available}\\
n &= \text{number of persons}\\
\vec{P^S_i} &= \text{Preference vector of job seeker $i$, } \in \mathbb{Z}^{+, m \times 1} \\
P^S &= [\vec{P^S_i} | \dots | \vec{P^S_n}] \in \mathbb{Z}^{+, m \times n}, \\ 
&\text{\indent Preference Matrix of Seekers} \\
\vec{P^O_i} &= \text{Preference vector of job owner $j$, } \in \mathbb{Z}^{+, n \times 1} \\
P^O &= [\vec{P^O_i} | \dots | \vec{P^O_n}] \in \mathbb{Z}^{+, n \times m}, \\ 
&\text{\indent Preference Matrix of Job Owners} \\
\vec{A} &= \text{Position Available vector }\in \mathbb{Z}^{+, m \times 1} \\
a_j &= \text{Amount of positions for job $j$, } \in \mathbb{Z}^{+} \\
X &= \text{Placement Matrix} \in \{0,1\}^{n \times m} \\
x_{i,j} &= \begin{cases}
1 & \text{if $S_i$ is slated for job $j$}\\
0 & \text{otherwise}
\end{cases} \\
C &= \text{Co-Location Matrix, upper triangular}\\
C_{ij} &= \begin{cases}
1 & j > i, \text{ and  Seeker $i$ requests co-location with Seeker $j$} \\
0 & j\leq i, \text{ or Seeker $e_{i,1}$ does not request co-location}
\end{cases} \\
n_c &= \text{number of couples requesting co-location}\\
&= \frac{1}{2}\sum_{i=1}^n\sum_{j=1}^n C_{ij}\\
\end{align}

