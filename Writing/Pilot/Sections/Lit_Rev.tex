% The most important people in this field, and the winners of the 2012 Nobel Prize in Economics for their work in stable marriage matching, are Roth \footnote{\url{http://stanford.edu/~alroth/PapersPDF.html}} and Shapely \footnote{\url{http://www.econ.ucla.edu/shapley/ShapleyBiblio.1.html}}.

\[1962\]

The intent of this algorithm is to provide stable pairings between job owners and job seekers based on their ranked preferences. The algorithm's initial conception and definition of stability can be found in Gale and Shapely's 1962 publication in the January \textit{The American Mathematical Monthly} \cite{gale_shapely}. The algorithm completes in polynomial time and was originally written for application in collage admissions.

\[1982\]

Roth explores the incentives of conveying true preferences and whether is it in everyone's best interest to do so. \cite{incentives} \textbf{working through the proofs to understand this better} In his work he specifically points to the applications to ``civil servants with civil service positions''.

\[1989\]

Irving explored indifference preferences and the follow on adoption for the Gale-Shapely algorithm. \cite{indifference} This provides the theoretical framework allowing for indifference in our own formulation. Though much of his focus is on differing forms of stability (weak, stongl, and super) these lie outside of our investigation due to the Navy's authority to compel its members to placement.

\[1993\]

Roth, Rothblum, and Vande Vate explored the concept of partial matches, discovering in fact this forms a lattice of solutions as well. These fractional matches could represent lotteries or time splitting. \cite{partial}

\[1994\]

Khuller et.al have developed an algorithm for stable matching on-line (matching people as they enter the system), as opposed to the typical formulation of having complete market participants and preferences at the time of matching. \cite{online} This could be interesting in future work of understanding the Navy detailing process as a continuous, rather than discrete, process.