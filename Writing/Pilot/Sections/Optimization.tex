\subsection{Optimization}

Alternate to deterministic algorithms is a linear programming (optimization) approach. This allows system owners to input strategic objectives in the seeker-owner job matching process. The importance of job seeker preference can be weighted to be more important than owner, or vice versa. Requirements can also be added, as you will see in the formulation below.

The optimization function takes the form:

\begin{align}
\min \qquad & \sum_{i = 1}^{i=n} \sum_{j = 1}^{j=m} f(x_{i,j}) \\
\text{ such that } \qquad & \sum_{j=1}^m x_{i,j} \leq 1 \quad  \forall i \in \{1, \dots n\} \quad \texttt{only one job per person} \\
& \sum_{i = 1}^{i=n} \sum_{j = 1}^{j=m}x_{i,j} == \min(n,m) \quad  \forall j \in \{1, \dots m\} \\& \quad \texttt{either all the jobs are filled or everyone has a job} \\
& \sum_{i=1}^n x_{i,j} \leq a_j \quad  \forall j \in \{1, \dots m\} \quad \texttt{all jobs are at or below capacity} \\
& \frac{1}{n_c} \sum_{i=1}^{n_e} C(e_1, e_2) \geq 0.95 \quad \texttt{at least 95\% of couples are co-located}
\end{align}

The Goodness Function $f$ is the strategic objective function of the assignment process. For the sake of this paper, we set it to value the preference of the seeker twice as much as the preference of th job owner.

\[f(x_{i,j}) = 2P^S_{i,j} + P^O_{j,i}\]

In matrix form this can be re-written (with $tr()$ indicating the trace):

\begin{align}
\min \qquad & tr(2X^TP^O) + tr(XP^S) \\
\text{ such that } \qquad & \sum_{j=1}^m x_{i,j} \leq 1 \quad  \forall i \in \{1, \dots n\} \quad \texttt{only one job per person} \\
& X^T \bullet 1 \leq A \quad \quad \texttt{all jobs are at or below capacity} \\
& \frac{1}{n_c} \sum_{i=1}^{n_e} X^TCXD \geq 0.95 \quad \texttt{at least 95\% of couples are co-located}
\end{align}


Co-location Function $C$ returns $1$ if the couple is considered co-located, 0 if not or if single. Here we choose 50 miles between job locations to be consider co-located because that is the threshold for receiving dislocation allowance (DLA) for a permanent change of station (PCS) according to the Joint Travel Regulations (JTR). The location function $L(S_i)$ returns the lat/long location of the stationing for Seeker $i$. 

\[C(e_1,e_2) = \begin{cases}
0 \quad & \text{ if } e_2 == 0 \\
 \mathbbm{1}\big( || L(S_i) - L(S_j) || \leq 50 \big) \quad & \text{ otherwise }
 \end{cases}\]


The inspiration and initial formulation of this optimization was done by a young Air Force officer who has since moved onto the private sector. 

The extension of the formulation to include the Co-Location Function $C$ and have more than one positions for each job $a_i$ are the contributions of this team. 

\subsubsection{Explanation of Co-location constraint math}

 \[\frac{1}{n_c} \sum \sum X^TCX \geq 0.95\]

 This works because

 \begin{align*}
(CX)_{ij} &= \begin{cases}
1 & \text{Seeker $i$'s mate is assigned to job $j$} \\
0& \text{otherwise}
\end{cases} \\
(X^T)_{ij} &= \begin{cases}
1 & \text{Seeker $i$ is assigned to job $j$} \\
0 & \text{otherwise}
\end{cases}\\
(X^TCX)_{ij} &= \begin{cases}
n & \text{Number of couples assigned to the $(i,j)$ job pairing} \\
0 & \text{otherwise}
\end{cases}\\
D_{ij} &= \begin{cases}
1 & \text{The job pairing $(i,j)$ is considered co-location } \\
0 & \text{otherwise}
\end{cases}\\
(X^TCX) \dot D &= \text{ The Hadamard (element-wise) multiplication of $(X^TCX)$ and $D$} \\
\big((X^TCX) \dot D \big)_{ij} &= \begin{cases}
n & \text{Number of co-located couples assigned to the $(i,j)$ job pairing} \\
0 & \text{otherwise}
\end{cases}\\
\sum \sum (X^TCX)_{ij} &= \text{Number of couples co-located} \\
\frac{1}{n_c} \sum \sum (X^TCX)_{ij} &= \text{Ratio of couples co-located}
 \end{align*}
