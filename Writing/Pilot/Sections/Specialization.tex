\[\textit{Definition}\] 
Specialization measures the extent to which the maximum desire for a sailor or job diverges from the mean. Sailor A is highly specialized if one job owner highly desires her much more than the mean job owner (e.g. Sailor A is the 1st preference choice of Job Owner X, but is on average, most job owners rank her as their 21st choice). In this example, Sailor A is a specialist to the extent that her skill set will serve a specific function well (with great benefit to Job Owner X). 

This comparative advantage over her peers represents a disproportionately positive contribution to the Navy by her being placed in that role compared to any other roles for which she may be qualified, the same as the importance of selecting her in relation to other more ‘competitive’ but less specialized candidates who could also work for Job Owner X. 

\[\texttt{Specialization}^S_i = \frac{\text{kurtosis}\{P^O_i\}}{\text{skew}\{P^O_i\}}\]

\[\textit{Impact}\] 
A sailor with a high specialization has some skill set that will serve a specific job function well; the sailor has an advantage over peers for a single role compared to the other roles for which the sailor is qualified. If most job owners rate a sailor with low preference but a single owner rates that sailor with a high preference, the marginal gains of the sailor being matched with that owner are high. Matching the sailor with this particular role utilizes her specialization and produces a disproportionate positive contribution to the Navy. 

Intuitively, the matching algorithm will favor creating high-specialization matches. Failing to make these matches has a more negative impact on the system than failing to make non-specialized matches, because the usefulness of the sailor goes down significantly between their optimal match and average matches. This metric may be useful in the future to examine if very particular aspects of jobs or sailors make them attractive to small subsets of the other population.
