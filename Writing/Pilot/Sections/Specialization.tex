\[\textit{Definition}\] 
Specialization measures the extent to which the maximum desire for a sailor or job differs from the mean. Sailor A is highly specialized if one job owner highly desires them much more than the mean job owner (e.g. Sailor A is the 1st preference choice of Job Owner X, but is on average the 21st ranked preference). A sailor is a specialist to the extent that their skill set will serve a specific function well, with a comparative advantage to their peers and with a disproportionate positive contribution to the Department of Defense compared to the other roles they are qualified for.

\[\texttt{Specialization}^S_i = \frac{\mu(P^O_i) - \min\{P^O_i\}}{\mu(P^O_i) + \min\{P^O_i\}}\]

\[\textit{Impact}\] 
A sailor with a high specialization has some skill set that will serve a specific job function well; the seeker has an advantage over peers for a single role compared to the other roles the seeker is qualified for. If most job owners rate a seeker with low preference but a single owner rates that seeker with a high preference, the marginal gains of that seeker being matched with that owner are high. Matching the sailor with this particular role utilizes their specialization and produces a disproportionate positive contribution to the Department of the Navy. Intuitively, the matching algorithm will favor creating high-specialization matches. Failing to make these matches has a more negative impact on the system than failing to make non-specialized matches, because the usefulness of the sailor goes down significantly between their optimal match and average matches. This metric may be useful in the future to examine if very particular aspects of jobs or sailors make them attractive to small subsets of the other population.