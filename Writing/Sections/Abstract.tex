\begin{abstract}

This paper seeks to outline the mathematics necessary to enable a detailing marketplace for the Department of Defense (DoD). A Detailing Marketplace is a system by which military members due to rotate (due to the mandate to change jobs every 1-3 years) can transparently rank their job preferences and the people who own those jobs can supply their preference of incoming personnel. This problem is rather unique to the military due to the captive market of many members obligated to remain in service due to contract, desiring to stay in for a pension, wishing to stay in through a sense of service, and an inability for lateral entry (almost all members needing to start from the bottom). 

This process is shared in some regards by the medical school graduates applying to the residency stage of their training. These graduates are applying to the pool of U.S. residency programs. The difference to college admissions is that the pool is also rather narrow, mostly coming from U.S. based medical schools, and the specificity of skill-set required for success is better understood. This similarity is why our initial matching algorithm (Gale-Shapely Deferred Acceptance Algorithm) is the same one used by this process, the National Residency Match Program. The Gale-Shapely algorithm's application to the residency matching problem earned the Nobel Prize in 2012.

This paper also proposes an optimization based solution to the matching process. The optimization is not found in the National Residency Match Program as medical students have the ability to reject their assigned position, a choice not always given to military members. Thus an optimization can often find a more optimal solution for the system (Department of Defense) at the expense of a few forced members.

Due to the importance of co-locating dual-military households (where two family members are in the military), we focus our matching algorithm and optimization proposals on solutions that would guarantee 95\% or greater co-location rate.

Acknowledging the difficulty of wrangling disparate and dated personnel data, this paper also explores helpful metrics that can be gleaned simply from submitted, ordered preferences by job seekers and job owners. These are competitiveness, similarity, generalism, and specialization. Interestingly, due to the fact that preferences are expressed on job seekers and the jobs themselves, these metrics can be developed about the jobs or the job seekers.

Further the paper ends with suggested metrics that would be gleaned if personnel data beyond preferences was accessible, clean, and structured. These include a similarity measure based on quality encodings and a suggested ordering of possible jobs or applicants. The latter is proposed to be enabled by deep learning (the underlying technology of Artificial Intelligence (AI)). The suggested ordering would not make decisions on placement, but rather provide job seekers and job owners with metrics distilling the vast amount of information about that which they intend the rank. The metric would aid them in the process of making their rankings.

The code to demonstrate the matching algorithms, optimization, and preference-based metrics can be found in Ian Shaw's Github Repository \texttt{Dynamic\_Manning}. \footnote{\url{https://github.com/ieshaw/Dynamic_Manning}}

\end{abstract}