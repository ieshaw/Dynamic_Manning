\subsection{Similarity}

To understand the multi-dimensional similarity of two vectors, the natural choice is the root-mean-square distance between the two vectors. Yet, a job seeker's first preference are much more important to them than their hundredth preference, indicating that the similarity of top preferences of two job seekers is indicative of their overall similarity than the comparison of their much lower ranked preference. Thus we propose a similarity measure of two job seekers using preference data should be a weighted root mean square distance between their two preference vectors.

\begin{align}
m &= \text{number of different jobs available}\\
n &= \text{number of persons}\\
\vec{S_i} &= \text{Preference vector of job seeker $i$, } \in \mathbb{Z}^{+, m \times 1} \\
\end{align}

Thus the similarity function is
\[\text{Similarity}(S_i, S_j) = \sqrt{\frac{1}{m} \sum_{k=1}^{m} \frac{2m - S_{i,k} - S_{j,k}}{2m}(S_{i,k} - S_{j,k})^2}\]

Keep in mind, this same construct can be used to provide a similarity score of jobs by substituting in the job owner's preference vectors.

