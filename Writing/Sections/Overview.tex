In this report we outline the mathematical basis for a detailing marketplace. We recognize that previous efforts have been attempted in this arena, many failing due to the inaccessibility of Navy personnel data. For this reason, most of this report focuses on what can be accomplished without personnel data, just the submitted preferences of job seekers and job owners.  Sections \ref{Matching} and \ref{Metrics} focus on efforts that would be enabled by only preferences, not until Section \ref{Beyond} do we allude to efforts that could be pursued if further data were available.

\subsection{Notation}

Throughout the paper, we will reference the notation listed in this section.

Consider a set of preferences $\{P^{x}_{y,z} \in \mathbb{Z}^+: x \in \{O,S\}, y \in \{1, \dots, n\}, z \in \{1, \dots, m\}\}$. This indicates the positive integer preference ranking of either the job owner or seeker ($O$ or $S$) for the $n$ available jobs and $m$ seekers. Also know that $l_n$ is the number of available positions in each job.\;

Consider a set of preferences $\{P^{x}_{y,z} \in \mathbb{Z}^+: x \in \{O,S\}, y \in \{1, \dots, n\}, z \in \{1, \dots, m\}\}$ and some of the pairs are submitted together $\{(P^{x}_{y_1,z}, P^{x}_{y_2,z}) \in \mathbb{Z}^+: x \in \{O,S\}, y_1,y_2 \in \{1, \dots, n\}, z \in \{1, \dots, m\}\}$, and indicate pairing by $p_y = 0$ if submitted as a single or $p_{y_1}=y_2, p_{y_2}=p_{y_1}$ if as a couple. To encapsulate the idea of seeking entities consider $\{A_q : q \in \{1, \dots, m_e\}\}$, where each entity $A_i$ can either be a single seeker $(A_i = y)$ or a pair, $(A_i = (y_1, y_2)$. This indicates the positive integer preference ranking of either the job owner or seeker ($O$ or $S$) for the $n$ available jobs and $m$ seekers. Also know that $l_n$ is the number of available positions in each job.\;


Consider our terms:

\begin{align}
m &= \text{number of different jobs available}\\
n &= \text{number of persons}\\
\vec{S_i} &= \text{Preference vector of job seeker $i$, } \in \mathbb{Z}^{+, m \times 1} \\
\end{align}



Consider these terms:

\begin{align}
m &= \text{number of different jobs available}\\
n &= \text{number of persons}\\
n_c &= \text{number of couples requesting co-location}\\
c_i &= \begin{cases}
r & \text{ Seeker $i$ requests co-location with Seeker $r$} \\
0 & \text{ Seeker $i$ does not request co-location}
\end{cases} \\
\vec{P^S_i} &= \text{Preference vector of job seeker $i$, } \in \mathbb{Z}^{+, m \times 1} \\
P^S &= [\vec{P^S_i} | \dots | \vec{P^S_n}] \in \mathbb{Z}^{+, m \times n}, \\ 
&\text{\indent Preference Matrix of Seekers} \\
\vec{P^O_i} &= \text{Preference vector of job owner $j$, } \in \mathbb{Z}^{+, n \times 1} \\
P^O &= [\vec{P^O_i} | \dots | \vec{P^O_n}] \in \mathbb{Z}^{+, n \times m}, \\ 
&\text{\indent Preference Matrix of Seekers} \\
\vec{a_j} &= \text{Amount of positions for job $j$, } \in \mathbb{Z}^{+} \\
X &= \text{Placement Matrix} \in \{0,1\}^{+, n \times m} \\
x_{i,j} &= \begin{cases}
1 & \text{if $S_i$ is slated for job $j$}\\
0 & \text{otherwise}
\end{cases}
\end{align}

