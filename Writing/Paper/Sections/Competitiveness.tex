\[\textit{Definition}\]

Competitiveness measures the relative desirability of a given sailor or job based on the expressed preference ranking of the other party. We wanted to consider a more sophisticated way of determining this metric beyond an average ranking for either the sailor or job as it is very susceptible to right tail bias.  

For example, a position ranked first by ten individuals and hundredth by twenty individuals has the same average ranking as a job ranked seventh by all thirty individuals; yet the former is much more competitive. To attempt to compensate for this potential discrepancy, we adjust the average by a power of one half as to lessen the impact of very low preferences, thereby weighting favorable preference more in our score consideration. 

In order to generalize the competitiveness score we scale the resulting average by the total number of jobs or sailors within the system. This allows for competitiveness to range from 1 being most competitive, and 0 being not competitive at all.


\begin{align}
F_j &= 1 - \frac{a_j}{M} \quad&&\texttt{ \% of openings job $j$ is of all openings in the Navy} \\ 
S^S_i &= 1 - \frac{1}{nm} \sum_j \sqrt{P_{ij}^O} \quad &&\texttt{ sought after metric} \\
C_i^S &= S^S_i \quad &&\texttt{ competitiveness of a job seeker} \\
C_j^O &= F_j S^O_j \quad  &&\texttt{ competitiveness of a job} \\
\end{align}

To follow the development of this metric, more details are in Appendix (\ref{Competitiveness_Extended}).

\[\textit{Impact}\]

Competitiveness will be used to rapidly identify the most attractive candidates (e.g. Sailor A has a 0.98 competitiveness score and is therefore a top ranked candidate amongst all job owners). These candidates would be good fits in many jobs and therefore could be considered ideal individuals to ‘screen for command’ -- the selection process for being a higher ranking commanding officer. A prominent stakeholder in our study, the Chief of Naval Personnel, has consistently sought after a way to rank sailors and evaluate their competitiveness as well. This is one way. 

On the job side, a low competitiveness would be a great way to decide on incentive structures. These jobs are those that very few sailors prefer and therefore ought to be incentivized to be filled, perhaps with the promise of promotion or some additional sort of compensation to target the right sailor to be willing to take an undesirable post. 
 