Under the backdrop of the President of the United States issuing the \textit{Executive Order on Maintaining American Leadership in Artificial Intelligence} in February 2019, we created the algorithmic foundation for the advancement of artificial intelligence and machine learning (AI/ML) in military talent allocation.  Before the US Government can accomplish its AI/ML goals, however, it must embrace the essential building blocks underlying the buzzwords. 

We believe the world to be arcing toward the need for humans to grow as increasingly specialized workers, and in the vicious games of statecraft and warfare, allocating talent with hyper specificity will become a necessity for victory. A building block on that path is the institutional acceptance of data-driven human decision making and algorithmic support of those human decisions. To that end, we theorized, coded, and prototyped an mixed integer programming (MIP) algorithm that matches job seekers (sailors) and job owners (commanding officers) more optimally than can any existing Department of Defense (DoD) process, algorithmic or human. 

The process of talent management through ‘detailing’ in the Navy is an arena ripe for such an MIP approach. Due to the structure of many sailors applying to jobs  with possibly more than one opening, this issue is a rendition of the college admission problem \cite{1962_Gale} without the need for stability \cite{1988_Roth_Sotomayor}. The essence of our solution is completed via algorithmic linear programing, specifically the use of MIP. We explore the nuances of how algorithms greatly increase optimal outcomes. Moreover, we discuss how MIP pushes optimality beyond deferred acceptance algorithms which, until now, were considered state-of-the-art for the DoD. 
