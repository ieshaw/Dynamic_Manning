The essential effort of this paper is to demonstrate the mathematical underpinnings of algorithmic matching for a Defense application. The pilot implementation of this algorithm, however, was a very human process. At the most foundational level, the algorithm solely requires preferences from job seekers (sailors) and job owners (commanding officers). In order for these preferences to be well informed, the system ought to have standardized job descriptions and sailor resumes.  Obtaining this central information proved to be the most difficult aspect of piloting the algorithm rather than the mathematics, no matter how novel the MIP approach or exquisite the introduction of implied preferences. 

Even Alvin Roth, who won a Nobel Prize for his foundational matching algorithm, experienced the same human and bureaucratic difficulties. Roth suggests, ``overall, one lesson from the [matching project] is that mechanism design in a political environment requires that not only policy makers themselves be persuaded of the virtues of a new design, but that they be able to explain and defend the mechanism to the various constituencies they serve.''
\cite{2006_Atila} The explainability of our algorithm is relatively simple in that it matches preferences from two sides -- seekers and owners -- yet any sort of deviation from existing processes creates concern and risk for senior stakeholders, real or imagined. Like building a boat in a bottle, the art of the prototype was not so much the algorithm itself, but the delicacy needed to create it within a debilitatingly constrained environment.

To best alleviate these issues, our team leveraged a web-based polling tool that allows both sailors and commanding officers to input their preferences. Sailors were given job descriptions and commanding officers were given sailor bios in order to inform their respective preferences.  In order to test it across larger data sets, however, we introduced a slew of sources described in the following section. 
