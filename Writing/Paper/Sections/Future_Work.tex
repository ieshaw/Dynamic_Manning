\subsection{Inspired from Academia}

Ongoing work should continue to explore the best ways in which markets can be created. This includes consideration of partial matches or lotteries in MIP.\cite{1993_Roth} One may also consider the ability to explore the viability of leveraging Budish's wagering formulation of approximate competitive equilibrium from equal incomes.  \cite{2011_Budish} Additional research would include how to incorporate synergy of selection preferences by Job owners if allowed more than one person. \cite{1985_Roth_b}  Further, there could be that service members could be place in multiple jobs, such as selecting their main role and their collateral duties in the same matching process. \cite{1982_Roth} Lastly, an open question the authors are curious about is the opportunity for unsupervised learning (such as K-means clustering) on the preference data to see if there are clusters of service members with regards to their preferences, and what theres clusters indicate.

\subsection{Asks from DoD Personnel}

From interviewing DoD leadership there are several institution specific asks. Many of our considered markets have an “on-line” placement procedure \cite{1994_Khuller}, such as that which happens at many information-sensitive commands due to the trickle of clearance issuance as opposed to bulk assignment. 

In the same spirit of temporal considerations, military members have set time-lines at each command and are given expected rotation dates. Consideration of overlapping rotation and end strength could give another direction to objective function formulation.

Sometimes military members have the option to take orders or separate from the service. A future formulation could allow service members to express ordinal preference up to a point, and then indicate if they do not receive any of those they would opt to separate from the service. A constraint, we would imagine, would need to be added with respect to retention on each detailing cycle.

Tailored Compensation decisions could be made for uncompetitive, unspecialized positions the DoD needs filled.

Explore strategic importance of positions. This can either be an iteration where billets in priority tranche's are run iterative (tier 1 billets all matched, those sailors and billets are taken out of the pool, then tier 2 billets all matched, etc.), or a weighting where in a single matching optimization the preferences of higher tiered billets are given a greater weighting.

Forthcoming work on verifying and validating our implementation of inferred preferences will continually be bolstered as well \cite{2019_Shaw}.