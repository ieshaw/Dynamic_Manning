Alternative formulations could additional consider tweaking the objective function to more directly apply to the features of a given market. As previously described, our work considered preference penalty in a squared function as the second pick for a sailor is not twice as bad as their fourth, but rather exponentially worse due to the intense nature of billet selection. Tweaking the objective function may include: multiple interior points, so space to tweak objective function.  \cite{1988_Roth_Sotomayor}

When building these markets, a key consideration not explored in this paper is the completeness or accuracy of the information from which the decision makers determine their preferences. Our interviews with implementers of matching markets revealed that the Army is currently advancing the ability for self professed skills and assessments, like a fitness test, to be a core part of the information environment. This can also include adding new data points such as having all junior officer taking the GRE as a part of their core training pipeline. while in the captain’s career course (CCC). This may additionally take the form of more accessible and data rich marketplace front-end systems that give job seekers and owner more clarity on the jobs or applicants they are considering. The Navy Explosive Ordnance Disposal community is leading the way in this domain with their Jetstream. We encourage the full-stack integration of our algorithmic back-end to Jetstream’s front-end infrastructure in order to have the most advanced talent marketplace in the entire DoD. 

An alternative formulation that came up frequently was coefficients or benefits for “diversification matches.” This is to say that job seekers who place high preferences on jobs that expand their skill sets could have lower coefficients, or lower burden to the objective function, over staying in a specialized domain. The opposite has comparative validity in some markets where matches should be prized over traditional career paths. The Army is already implementing this concept. NDAA 19-503/4/ is the law that permits such specialization, and the Army is currently writing policy for ‘brebet’ promotions. 
