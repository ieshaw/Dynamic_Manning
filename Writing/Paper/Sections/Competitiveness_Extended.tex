\subsubsection{Average Ranking}

The competitiveness score for any given job could be defined as the average preference ranking of the job across all seekers.

\[\texttt{Competitiveness}_j = \frac{1}{n} \sum_{i=1}^m P^S_{i,j}\]

\subsubsection{Adapted Sciorintino Ratio}

The issue with the \textit{Average Ranking} approach is that, though conveying the desirability of a position, loses much of the information contained in the distribution of preferences. Consider two jobs out of a pool of ten jobs with five applicants.

\begin{align}
O_1 &= [1,1,1,1,5] \\
C_1 &= 1.8\\
O_2 &= [2,2,2,2,1] \\
C_2 &= 1.8
\end{align}

We see that though job 1 is more competitive, job 2 has the same score, thus considered metrically, equally competitive.

An average of preference rankings, . For example, a position ranked first by ten individuals and hundredth by a twenty individuals has the same average ranking as a job ranked seventh by all thirty individuals; yet the prior is much more competitive. 

The Sciorintino Ratio is a metric from the finance industry used to measure the performance of a investment vehicle based on the distribution of their returns. An investment vehicle with consistent, small positive returns is much different than one with consistent negative returns and one big win, even though the average return may be the same. We see the same problem in finance as the detailing marketplace, conveying the nature of a distribution in a single metric. The difference in this ratio, as compared to the more popular Sharpe Ratio, is that the Sciorintino ratio does not let positive volatility negatively effect the score, only volatility on the negative side is punishing. Here we want to adapt the the concept that the volatility of preference is reflected only with higher ranked preferences. For example, if a job has an average ranking of seven, the fact that many people ranked it first is much more important for competitiveness than the fact that many other people ranked it hundredth.

The formulation for our \textit{Adapted Sciorintino Ratio} for the competitiveness of job $j$ takes the form

\begin{align}
\mu_j &= \frac{1}{n} \sum_{i=1}^m P^S_{i,j} \\
\sigma_j &= \frac{1}{m^2} \sum_{i=1}^m \mathbbm{1}\big( P^S_{i,j} < \mu \big) \big(P^S_{i,j} - \mu \big)^2 \\
\texttt{Competitiveness}_j &= \frac{\mu_j}{\sigma_j}
\end{align}

\subsubsection{Weighted Scaling}

The issue with the \textit{Adapted Sciorintino Ratio} approach is that it has trouble adjusting for consistent ranking around the mean. Consider two jobs out of a pool of ten jobs with five applicants.

\begin{align}
O_3 &= [3,3,3,2,1] \\
C_3 &= 56\\
O_4 &= [4,3,4,2,1] \\
C_4 &= 24
\end{align}

We see that though job 1 is more competitive, job 2 has a lower score, thus considered metrically more competitive.

So now we turn to a different ranking that scales based on the number of people, the number of jobs available, and the number of positions available in each job. This job scales from 1 being most competitive, and 0 being not competitive at all.

\[\texttt{Competitiveness}_j = 1 - \frac{1}{mn \sqrt{a_j}} \sum_{i=1}^n \sqrt{P^S_{i,j}}\]

Applying this to the previous example (assuming each has only one position available) we see the scores are in the proper order (job 2 is less competitive than job 1).

\begin{align}
O_3 &= [3,3,3,2,1] \\
C_3 &= 0.912\\
O_4 &= [4,3,4,2,1] \\
C_4 &= 0.906
\end{align}

Yet we see that this is just the mean ranking squared scaled by the number of jobs available and the number of positions for the job. So, this scoring method still suffers from the issue of the average ranking, heavy tailed distribution of preferences may have the same competitiveness score as a normal distribution, even though the former is more competitive. To attempt to compensate for this, we adjust the average by a power of one half is to lessen the impact of preferences closer to $m$ on the score, essentially weighting favorable preference more in our score consideration.

This can be seen by returning to our first example.

\begin{align}
O_1 &= [1,1,1,1,5] \\
C_1 &= 0.892\\
O_2 &= [2,2,2,2,1] \\
C_2 &= 0.885
\end{align}

\subsubsection{Weighted Scaling: Group Formulation}

\begin{align*}
F_j &= 1 - \frac{a_j}{m_a} \\
&= \text{ \% of openings job $j$ is of all openings in the Navy} \\ 
S^S_i &= 1 - \frac{1}{nm} \sum_j \sqrt{P_{ij}^O} \\
&= \text{ sought after metric} \\
C_i^S &= S^S_i \\
&= \text{ competitiveness of a job seeker} \\
C_j^O &= F_j S^O_j \\
&= \text{ competitiveness of a job} \\
\end{align*}

%\textbf{TODO: put an example in here.}