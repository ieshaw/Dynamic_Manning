The Department of Defense maintains a consistent matching problem in the placement of military members into jobs due to the requirement that uniformed service members rotate positions every three years. Gale and Shapely provided a stable matching process 1962 \cite{1962_Gale} that if applied to this problem would greatly improve the current, manual matching process. In this paper we go a step further noting that mixed-integer programming can be applied in the matching process to sacrifice stability in favor of military leadership goals. Additionally, we glean novel metrics from the expressed ordered preferences of both service members for jobs as well as job owners’ preferences for service members. These metrics are: specialization, competitiveness, similarity, and preference correlation. We formulated our prototype for the Chief of Naval Personnel. Nevertheless, our work is agnostic and can be applied not only to any service branch, but to any matching market not requiring stability, defined by obliged market participants.  \footnote{The code to demonstrate the matching algorithms, optimization, and preference-based metrics can be found in Ian Shaw's \texttt{Dynamic\_Manning} Github Repository . \url{https://github.com/ieshaw/Dynamic_Manning}}
