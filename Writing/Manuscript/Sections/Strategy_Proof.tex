An important note is the attempt at strategy proofness of the algorithm \citep{2011_Budish} \citep{2006_Atila}, thus creating an incentive for honesty \citep{1982_Roth}. Proper market construction ensures that there is no benefit for a participant to provide incomplete preferences.  Likewise, we must acknowledge that there can be difficulty associated with a market participant ranking an incredibly large number of opportunities. To meet this need, we complete any incompletely provided preferences using the `implied preference'  method \citep{2019_Shaw}. 

The complete strategy proofness investigation of this formulation is left as an aspect of future work, but we acknowledge that there could be benefits to coalitions or awareness of the preference landscape of competitors. There is also a possibility of rejecting the implied preferences, as one market maker concluded unexpressed preferences indicate indifference \citep{1994_Irving} -- an expression of preference all its own. Nevertheless, the strategy proofness of Irving’s approach in an MIP matching process has yet to be explored. In this same vein, the aformentioned 'implied preference' method has not been expanded to to encompass the `separation' preference described in Section \ref{Retention}. 
