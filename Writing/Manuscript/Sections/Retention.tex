Retention is often and important goal of military leaders wanting to maintain manpower numbers in order to keep the force ready for possible operations. The difficulty is, with an all volunteer force, if a service member is beyond contractual obligation, they need to have incentive to stay in. For example, a sailor may be willing to stay in for another 3 years if they are guaranteed to stay in San Diego so that their child can finish out high school there. If asked to move across the countryi, then they will opt to leave the service and seek local, civilian employment. 

For the purpose of maximizing retention (minimization is just a sign change), we can have the objective function (\ref{obj_func}) take the form below.

\begin{align*}
f(X) &= \gamma\left(\sum_{i=1}^n \sum_{j=1}^m X_{ij}(P^S_{ij} + P^O_{ji})\right) + \lambda (R(X)) \\
P^S_j &= \begin{cases}
m + 1 &  \text{Would rather separate than accept assignment to } j \\
[1,m] \cap \mathbb{Z}  &  \text{otherwise expressed preference} \\
\end{cases} \\ 
\gamma, \lambda &= \texttt{weighting coefficients}, \in [0,1] \cap \mathbb{R} \\
R(X) &= \texttt{Retention rate of assignment set }X \\
R(X) &= \frac{\sum_{i=1}^n \mathbf{1}\left(X\bullet P^S\right < m+1)}{n}
\end{align*}

We can also incorpoate retention as a constraint. Say a service leader were to desire retention above 90\%, then the following constraint would be necessary. 

\[R(X) \geq \texttt{Desired retention rate between 0 and 1} \]

Keep in mind, there is maximum rate on retention that is not necessarily 100\%, therefore when employing this constraint special care must be taken to guarantee feasibility.  

The maximum retention rate given the constraints of the college admission formulation (defined in lines (\ref{one_job}, \ref{all_filled}, \ref{capacity})), is the value of the objective function for the program described below. Keep in mind these constraints must be feasible for this analysis to be worthwhile; this requires that there is at least one solution where there are is at least one job seeker willing to fill each position in the market.

\begin{align*}
\max \qquad & R(X) \\
\text{ such that } \qquad & \sum_{j=1}^m X_{i,j} \leq 1 \quad  \forall i \in \{1, \dots n\} \\
& \sum_{i = 1}^{i=n} \sum_{j = 1}^{m}X_{ij} = \min \left(n,\sum_{j = 1}^{m}a_j \right) \quad  \forall j \in \{1, \dots m\}  \\
& \sum_{i=1}^n X_{ij} \leq a_j \quad  \forall j \in \{1, \dots m\}
\end{align*} 

We suspect the implementation of these retention in the objective function and/or constraints to be strategy proof (indicating it is in the best interest of market participants to honestly express their ordinal preferences) so long as the placements described in the output are enforced (as opposed to used for negotiation). Yet, we have not explored the strategy proofness rigorously and leave that investigation to future work.
