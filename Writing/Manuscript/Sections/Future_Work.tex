\subsection{Inspired from Academia}

Ongoing work should continue to explore the best ways in which markets can be created. This includes consideration of partial matches or lotteries in MIP.\citep{1993_Roth} One may also consider the ability to explore the viability of leveraging Budish's wagering formulation of approximate competitive equilibrium from equal incomes.  \citep{2011_Budish} Additional research would include how to incorporate synergy of selection preferences by Job owners if allowed more than one person. \citep{1985_Roth_b}  Furthermore, service members could be placed in multiple jobs, such as selecting their main role and their collateral duties in the same matching process. \citep{1982_Roth} Lastly, an open question the authors are curious about is the opportunity for unsupervised learning (such as K-means clustering) on the preference data to see if there are clusters of service members with regards to their preferences, and what theres clusters indicate.

\subsection{Requests from DoD Personnel}

From interviewing DoD leadership there are several institution specific asks. Many of our considered markets have an “on-line” placement procedure \citep{1994_Khuller}, such as that which happens at many information-sensitive commands due to the trickle of clearance issuance as opposed to bulk assignment. 

In the same spirit of temporal considerations, military members have set time-lines at each command and are given expected rotation dates. Consideration of overlapping rotation and end strength could give another direction to objective function formulation.

Tailored Compensation decisions could be made for uncompetitive positions the DoD needs filled.

%Forthcoming work on verifying and validating our implementation of inferred preferences will continually be bolstered as well \citep{2019_Shaw}.
