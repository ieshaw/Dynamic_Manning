Our solution to the talent allocation problem is to optimize the matching of preferences from both sides: job owners (commanding officers) and job seekers (sailors). This general similarity between the military and medical  markets is the reason our investigation builds off the Gale-Shapley deferred acceptance algorithm used in the National Residency Match Program \citep{1962_Gale}. 

This paper proposes an optimization-based solution to the matching process. The optimization is not found in the National Residency Match Program due to the need for stability. Stability here is defined as a system where no two individuals from opposite sides of a market (male and female in the stable marriage problem) can leave their assigned partners and be better off for it. Stability is necessary for  medical students who have the ability to reject their assigned position, a choice not available to most military members. Thus preference matching in the military context does not require stability. Unburdened by the constraint of stability, we can leverage mathematical optimization \citep{1984_Roth} \citep{1985_Roth_b} \citep{1989_Roth}; this insight is the key to the rest of our investigation. 

We found that a traditional algorithm, like Gale-Shapley deferred acceptance, was unconducive to the creation of modular constraints that common in US Navy personnel policy. One such constraint is the Chief of Naval Personnel's goal to have at least 95\% of dual-military couples stationed within 50 miles of their partner. To gain modularity we turn to binary-integer programming.  We later show that this approach can outperform the deferred acceptance iteration in our desired metrics, in addition to its ease of introducing constraints without massive re-coding. 
