The below metrics are how  military leaders usually describe placement performance. Not uncommon is the announcement after service selection at the US Naval Academy, ``This year XX midshipman got their first choice, YY in their top five, and an average preference assignment of ZZ.''

\subsection{Preference Allocation Windows}
\label{windows}
Preference allocation indicates how many individuals received their top preference, how many individuals got a top one, top five, and top ten preference, and how many failed to be matched with any preference.

\subsection{Average Preference Sum}
\label{pref_sum}
Preference sum is the addition of the ordinal preference of a sailor for their assigned command with the ordinal preference of the command for their assigned sailor. For example if a sailor is assigned their most preferred command but the command ranked the sailor third, the preference sum would have a value of 4. The average of this sum accross the market provides a high level understanding of how the matching system performs in aggregate.

% \[\textit{Talent Distribution}\]
% Some job owners have multiple jobs. Ideally, such job owners would get an approximately equal set of preferences; one owner with five job slots would not get their first through fifth choices while another job owner with five job slots gets their 20th-25th preferences. 

% \[\textit{Preference Difference Gap}\]
% This metric helps to illuminate the separation between job owner and sailor preferences. This can be an indicator for senior leaders that neither sailors nor job owners had their preferences unduly weighted within the optimization system.  
