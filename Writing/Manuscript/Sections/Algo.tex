A basic algorithm, like Gale-Shapley’s set of instructions for stable marriages, is an important first step in talent allocation. We propose an application of binary optimization, due to their adaptability to shifting objective and constraints.

The constraints we apply are the same as the college admissions problem \cite{1985_Roth}, where each job can fill up to but not exceed the allocated number of positions and each sailor can only be assigned to one position with the added constraint that the solution is as good or better than deferred acceptance when compared by the post-match metrics in Section \ref{post-match}. We chose this to show how this approach can improve upon deferred acceptance and to guarantee feasibility (since the deferred acceptance output is a feasible solution) the explicit program is in Section\ref{Matching}. Though a comparison to deferred acceptance is of academic interest, we also provide alternate formulations that may be of more pragmatic interest to military leaders. These include co-location (Section \ref{CoLocation}), quality spread (Section \ref{QualitySpread}), weighting the importance of certain command's preferences over others (Section \ref{Importance}), and retention (Section \ref{Retention}).

