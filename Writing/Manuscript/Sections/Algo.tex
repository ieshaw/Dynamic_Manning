The constraints we apply are the same as the college admissions problem \citep{1985_Roth}. The line numbers refer to their mathematical formulation in the binary-integer program of Section \ref{matching}.
\begin{enumerate}
\item Each sailor can only be assigned to one position (\ref{one_job})
\item Either all the jobs are filled or every job seeker is placed (\ref{all_filled})
\item Each job can fill up to but not exceed the allocated number of positions (\ref{capacity})
\end{enumerate}

For the sake of the prototype improving upon deferred acceptance in our desired metrics (explained in Section \ref{windows}) we also include the constraints in line (\ref{s_da}, \ref{o_da}). This constellation of constraints has the added benefit of guaranteed feasibility since the deferred acceptance output is a feasible solution.

Though a comparison to deferred acceptance is of academic interest, we also provide alternate formulations that may be of more pragmatic interest to military leaders. These include co-location (Section \ref{CoLocation}), quality spread (Section \ref{QualitySpread}), weighting the importance of certain command's preferences over others (Section \ref{Importance}), and retention (Section \ref{Retention}).
