Human talent allocation is a resource-intensive process for an institution. This paper outlines the mathematics necessary to lay the groundwork for such a task within one of the most complex institutions of American society -- the DoD.  We dove into this problem seeking to show the validity of the mathematics, as well as to discover the challenges that exist when creating a talent marketplace from the ground up. Furthermore, despite the small prototype sample, the find the algorithm to be widely scalable, particularly if combined with a proper front-end as our project is the back-end of a future full-stack system. 

Market forces are largely the drivers behind a job marketplace -- this concept is not new for the American economy.  However, given a captive audience such as service members ordered to specific jobs, the mathematical benefit of a ‘benevolent autocracy’ is immense.  Rather than the suboptimal fate to which current servicemembers are subjected, the DoD is already shopping for a better way forward algorithmically.  We improve upon what currently exists in all metrics except current implementation; nevertheless, we have a plan with stakeholders to quickly field the technology. 

In the Navy, terminology of a ‘detailing marketplace’ refers to the attempts to ‘detail,’ or place, warfighters into their next job.  This detailing marketplace is a system by which military members can transparently rank their job preferences and the people who ‘own’ those jobs can supply their preference of incoming personnel. This problem is rather unique to the military due to the aforementioned captive market of many members obligated to remain in service due to contract, desiring to stay in for a pension, or perhaps wishing to stay due to a sense of public service. This market type is complemented by an inability for lateral entry -- almost all members need to start from the entry level.  Additionally, servicemembers typically change jobs every one to three years. 

There are ontological parallels to other communities, too. This process is shared in some regards by the medical school graduates applying to the residency stage of their training. Those graduates apply to a pool of US residency programs. This pool is also rather narrow, mostly coming from US based medical schools, and the specificity of skill set required for success is better understood and measured than success as a military officer. 

