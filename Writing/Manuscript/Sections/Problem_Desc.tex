Human talent allocation is a resource-intensive process for an institution. This paper outlines the mathematics necessary to lay the groundwork for such a task within one of the most complex institutions of American society -- the DoD.  We dove into this problem seeking to show the validity of the mathematics, as well as to discover the challenges that exist when creating a talent marketplace from the ground up. Furthermore, despite the small prototype sample, the find the algorithm to be widely scalable, particularly if combined with a proper front-end as our project is the back-end of a future full-stack system. 

Market forces are largely the drivers behind a job marketplace -- this concept is not new for the American economy.  However, given a captive audience such as service members ordered to specific jobs, the mathematical benefit of a ‘benevolent autocracy’ is immense.  Rather than the suboptimal fate to which current servicemembers are subjected, the DoD is already shopping for a better way forward algorithmically.  We improve upon what currently exists in all metrics except current implementation; nevertheless, we have a plan with stakeholders to quickly field the technology. 

In the Navy, terminology of a ‘detailing marketplace’ refers to the attempts to ‘detail,’ or place, warfighters into their next job.  This detailing marketplace is a system by which military members can transparently rank their job preferences and the people who ‘own’ those jobs can supply their preference of incoming personnel. This problem is rather unique to the military due to the aforementioned captive market of many members obligated to remain in service due to contract, desiring to stay in for a pension, or perhaps wishing to stay due to a sense of public service. This market type is complemented by an inability for lateral entry -- almost all members need to start from the entry level.  Additionally, servicemembers typically change jobs every one to three years. 

There are ontological parallels to other communities, too. This process is shared in some regards by the medical school graduates applying to the residency stage of their training. Those graduates apply to a pool of US residency programs. This pool is also rather narrow, mostly coming from US based medical schools, and the specificity of skill set required for success is better understood and measured than success as a military officer. 

This general similarity between markets is the reason our initial matching algorithm (Gale-Shapley deferred acceptance algorithm) resembled that of the National Residency Match Program. For context, the Gale-Shapley algorithm's application to the residency matching problem earned the 2012 Nobel Prize in Economic Sciences.

This paper also proposes an optimization-based solution to the matching process. The optimization is not found in the National Residency Match Program due to the need for ‘stability.’ Stability here is exemplified by medical students’ ability to reject their assigned position, a choice rarely given to military members, if at all. Thus, optimization in the military context does not require stability.  As a result, mathematical optimization, unburdened \cite{1984_Roth} \cite{1985_Roth_b} \cite{1989_Roth} by the luxury of stability, is at its peak in our formulation -- a notable improvement. 
We found that a traditional algorithm, like Gale-Shapley deferred acceptance, was unconducive to the creation of modular constraints.  A constraint may be the introduction of a rule based on Navy policy that the algorithm would have to satisfy before conducting all matches (e.g. at least 95\% of  dual-military couples ought to be stationed within 50 miles of their partner). For that reason, we transitioned to mixed integer programming, a more agile linear programming approach to the problem.  We later show that MIP outperforms the deferred acceptance iteration outright, in addition to its ease of introducing constraints without massive re-coding. 

Acknowledging the difficulty of wrangling disparate and dated personnel data, this paper also explores helpful metrics that can be gleaned simply from submitted, ordered preferences by job seekers and job owners. These are specialization, competitiveness, similarity, and preference correlation. These metrics can be used to describe both sides of the market, on the seeker or the owner alike. 
 
Further, the paper ends with suggested metrics that would be gleaned if personnel data beyond mere preferences were accessible, clean, and structured. These include a similarity measure based on quality encodings and a suggested ordering of possible jobs or applicants. AI/ML, specifically deep learning, could enable the latter. The suggested ordering would not make decisions on placement, but rather provide job seekers and job owners with an intelligent ordering, allowing a human participant to more easily distill the vast amount of job information relevant to her. 

