There is precedent in the DoD to do assignments in a manner by which the distribution of quality officers is even accross some sort of grouping. This is the assignment policy for US Navy submariners to boats after initial training and US Marines to MArine Occupational Spectialties (MOS). The purpose of this is to ensure that no one unit or community within the force falls behind in competency due to a preponderance of low-performing personnel. 

To incorporate this into the matching algorithm would be easy. We would simply make the below formulation the objective function at (\ref{obj_func}).

\[ f(X) = \gamma \sum_{i=1}^n \sum_{j=1}^m X_{ij} \left(P^S_{ij} + P^O-{ji}\right) + \lambda (\sigma^2(X)) \]

The explanation of the objective function can be found below.

\begin{align*}
f(X) &= \gamma \sum_{i=1}^n \sum_{j=1}^m X_{ij} \left(P^S_{ij} + P^O-{ji}\right) + \lambda (\sigma^2(X)) \\
    \gamma, \lambda &= \texttt{weighting coefficient}, \in [0,1] \cap \mathbb{R} \\
    \sigma^2(X) &= \texttt{Variance of the quality assignments} \\
    \sigma^2(X) &= \frac{\sum_{j=1}^m (\mu^q_j - \bar{\mu^q})^2}{m-1} \\
    \mu^q_j &= \texttt{Average Quality score of person assigned to position } j \\
    \mu^q_j &= \sum_{i=1}^n  \frac{X_{ij} Q_i}{A_j} \\
    \bar{\mu^q} &= \sum_{j=1}^m \frac{\mu^q_j}{m} \\
    Q &= \texttt{Quality scores for each seeker }, \mathbb{R}^{n \times 1}
\end{align*}
